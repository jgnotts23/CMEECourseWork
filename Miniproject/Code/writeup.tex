\documentclass[twoside,twocolumn]{article}

\usepackage[sc]{mathpazo} 
\usepackage[T1]{fontenc} 
\linespread{2} 
\usepackage{microtype} 
\usepackage[english]{babel} 
\usepackage{natbib}
\usepackage[switch]{lineno} 

\usepackage[hmarginratio=1:1,top=32mm,columnsep=20pt]{geometry} 
\usepackage[hang, small,labelfont=bf,up,textfont=it,up]{caption} 
\usepackage{booktabs} 

\usepackage{lettrine} 

\usepackage{enumitem} 
\setlist[itemize]{noitemsep} 

\usepackage{abstract} 
\renewcommand{\abstractnamefont}{\normalfont\bfseries} 
\renewcommand{\abstracttextfont}{\normalfont\small\itshape} 

\usepackage{titlesec} 
\renewcommand\thesection{\Roman{section}} 
\renewcommand\thesubsection{\roman{subsection}} 
\titleformat{\section}[block]{\large\scshape\centering}{\thesection.}{1em}{} 
\titleformat{\subsection}[block]{\large}{\thesubsection.}{1em}{}

\usepackage{fancyhdr}
\pagestyle{fancy}
\fancyhead{} 
\fancyfoot{} 
\fancyhead[C]{Miniproject $\bullet$ March 2019 $\bullet$ J. Griffiths} % Custom header text
\fancyfoot[RO,LE]{\thepage} 

\usepackage{titling} 

\usepackage{hyperref} 


\setlength{\droptitle}{-4\baselineskip} 

\pretitle{\begin{center}\Huge\bfseries} 
\posttitle{\end{center}} 
\title{Modelling Thermal Responses of Metabolic Traits} 
\author{
\textsc{Jacob Griffiths} \\
\normalsize Imperial College London \\ 
\normalsize \href{mailto:jacob.griffiths18@imperial.ac.uk}{jacob.griffiths18@imperial.ac.uk}
}
\date{\today} 
\renewcommand{\maketitlehookd}{

\begin{abstract}
\noindent Hello this is the Abstract
\end{abstract}
}

\begin{document}

\maketitle

\linenumbers
\section{Introduction}

\lettrine[nindent=0em,lines=3]{T}emperature is a fundamental parameter in
almost all processes and its importance in metabolic biology is well documented \citep{Montoya2012, Dell2011,DeLong2017}. 
Metabolic processes are catalysed by enzymes
which depend on kinetic, and ultimately heat, energy to function. As temperature
decreases, atoms and molecules move progressively slower and thus metabolic rates
decrease accordingly whereas when temperature increases, metabolic rates increase rapidly
until the thermal optimum is reached  \citep{DeLong2017, Dell2011}. That is, the temperature at which optimal
metabolic rate occurs ($T_{pk}$ or $T_{opt}$). Beyond this, increasing temperature
will start to hinder the metabolic rate as the proteins that enzymes are composed of 
will start to denature until the process stops entirely. These metabolic responses 
to temperature exhibit a remarkably similar pattern, often referred to as Thermal 
Performance Curves (TPCs), across an array of metabolic processes and taxa. This 
makes the study of TPCs a useful tool of comparison for all life on Earth as all
species depend on metabolism for their energy. Furthermore, an increased understanding
of how species respond to temperature is imperitive in a rapidly warming world. 
If we can find some plasticity in a species' temperature tolerance then perhaps it will
have a better chance of avoiding the mass extinction that is sweeping our planet,
although some recent findings suggest the scope for adaptation may be limited \citep{Tuzun2018}.
Of course, there are other ways a species may adapt through latitudinal range shifting 
or evolution, but the latter seems unlikely in the time-frame available and the former
is only possible for mobile species with suitable habitats to move to.

\subsection{Models}
Three models were used in this study to compare their ability to fit to each dataset within BioTraits.
Firstly, the cubic polynomial was used as a phenomenological model with the following form:
\begin{equation}
B = B_0 + B_1T + B_2T^2 + B_3T^3
\end{equation}
Where $B$ is the responding trait value and $T$ is the temperature.
Secondly, the Briere model \citep{Briere1999} was used as an alternative phenomenological model:
\begin{equation}
  B = B_0T(T - T_{0})\sqrt{T_{m} - T}
\end{equation}
Where $T_{0}$ and $T_{m}$ are the minimum and maximum tolerances for the trait, $B$, and $B_0$ 
is a normalisation constant. Whilst this model is phenomenological, it can still provide useful 
biological information when fit as it provides an estimate of the minimum and maximum thermal 
tolerances of a particular trait for a particular organism. However, it falls short of the definition
of a mechanistic model as the model provides no insight into the underlying biological mechanisms at work.
Therefore, the third model used in this study was a simplified version of the Sharpe-Schoolfield \citep{Schoolfield1981}
model to provide a mechanistic comparison as it was formulated from thermodynamic and enzyme 
kinetic theory. The full model is given by:
\begin{equation}
  B = \frac{B_{0}e^{\frac{-E}{k}(\frac{1}{T} - \frac{1}{283.15})}}{1 + e^{\frac{E_l}{k}(\frac{1}{T_l} - \frac{1}{T})} + e^{\frac{E_h}{k}(\frac{1}{T_h} - \frac{1}{T})}}
\end{equation}
And the simplified:
\begin{equation}
  B = \frac{B_{0}e^{\frac{-E}{k}(\frac{1}{T} - \frac{1}{283.15})}}{1 + e^{\frac{E_h}{k}(\frac{1}{T_h} - \frac{1}{T})}}
\end{equation}
Where $k$ is the Boltzmann constant ($8.617 \times 10^{-5}  \ $eV$ \ \dot K^{-1}$), $B_0$ is the trait value at a 
reference temperature (283.15 K in this study), $E_l$ is the low-temperature deactivation energy (eV) of the enzyme and controls
the behaviour of the curve at very low temperatures and $T_l$ is the temperature at which 50\% of the enzyme is low-temperature deactivated.
$E_h$ is the high-temperature deactivation energy of the enzyme and controls the behaviour of the curve at high temperatures and $T_h$ is the 
temperature at which 50\% of the enzyme is high-temperature deactivated. $E$ is the activation energy which controls the behaviour of the curve 
in the enzyme's 'normal operating range', that is before $T_pk$ but not at low temperatures. The simplified version was chosen for this study
as low-temperature deactivation is weak and a lot of datasets within BioTraits lacked sufficient data at low temperatures. It also allows for 
more datasets to be used as the minimum number of datapoints required for the six-parameter full model would be larger than for the four-parameter
simplified version.



\section{Methods}

\subsection{Data}
The database used in this study, BioTraits, was provided by my supervisor,
Dr. Samraat Pawar and is an extension of the database used by \cite{Dell2011}. It consists of 
2165 unique thermal responses of metabolic 
processes from 1010 publications. Predominantly, respiration, growth and photosynthetic
rate are the metabolic process being measured against temperature. BioTraits includes species from many 
Phyla and with diverse life histories but a majority of representitives are terrestrial
species, often Arthropods. 
\\
\\
As this dataset contains 155 columns and 25826 rows, it was first refined to a handful
of relevant columns for this study to improve computational speed, namely the trait value and temperature.
Rows with missing values for these columns were removed and any sub-dataset with less than five 
datapoints was removed as this is the minimum required to estimate four parameter models like the cubic
polynomial and simplified Schoolfield. 

\subsection{Parameter estimation}
Using R, starting parameters for every unique sub-dataset in BioTraits were calcualted.
For the cubic polynomial model, starting parameters of 1 were used for all four parameters.
For Briere, estimates for $T_{0}$ and $T_{m}$ were made using
the minimum and maximum recorded temperatures respectively. For Schoolfield, A reference temperature of 10 
degrees Celsius (283.15 K) was used for as this has been used effectively in previous publications \citep{Dell2011} and $B_{0}$ 
was estimated as the recorded trait value nearest to this temperature, by definition. The peak metabolic rate $B_{max}$ was then
calculated ($T_{pk}$ being the corresponding temperature at this trait value) and the dataset was split around this value. If $B_{max}$ occurred at the highest recorded temperature 
(i.e. the rate had not started descending yet) the dataset was not split and the following regression was carried 
out on the whole dataset. The trait values of each side were logged and the reciprocals
of the temperature values were multiplied by the boltzmann constant ($8.617 \times 10^{-5} eV \cdot K^{-1}$). Linear 
regression was carried out on the left-hand (below $T_{pk}$) data and the estimate for $E$ was taken to be the gradient
of this line, with the $Eh$ estimate being twice this value. If regression failed, default estimates of $0.65$ for $E$
and $1.3$ for $Eh$ were used as recommended defaults from the literature and $E$ was given bounds of 0 to 3 while $Eh$ was bounded between
0 and 6 \citep{Montoya2012, Dell2011}. $Th$ was estimated by 
calculating the nearest recorded temperature to $B_{max} / 2$ as $Th$ is the temperature at which half the enzyme 
units have been made inactive so this provides a good estimate, with a lower bound of $T_{pk}$ and an upper bound of 400 Kelvin applied 
\citep{Sal2018}. For datasets with no datapoints after $B_{max}$, $Th$ was given a starting value equal to $T_{max}$. 

\subsection{Model comparison}
The Akaike Information Criterion (AIC) \citep{A1974} was used to compare model fits within each dataset and is given by the formula assuming the 
model is univariate, is linear in its parameters and has normally-distributed residuals:
\begin{equation}
  AIC = 2k - 2\ln(\hat{L})
\end{equation}
Where $\hat{L}$ is the maximum likelihood estimation and $k$ is the number of parameters for the model.
This method was chosen as it rewards the relative goodness of fit between models on the same data but penalises number of parameters used 
as this can sometimes lead to overfitting. However, despite this penalty, AIC can still be prone to favouring models with more parameters if 
the sample size is small. This can be circumvented by using AICc \citep{Hurvich1989}, an extension of AIC given by:
\begin{equation}
  AICc = AIC + \frac{2k^2 + 2k}{n-k-1}
\end{equation}
Where $n$ is the sample size and $k$ is the number of parameters as before. It should be noted that as $n \to \infty$, the additional parameter
penalty tends to zero and thus AICc tends to AIC, making it suitable for large samples too. As some of the datasets used in this analysis had only a handful of datapoints,
AICc was used to compare models instead of AIC.
\\
In addition to AICc, adjusted $R^2$, or $\bar{R}^2$, was used as an alternative comparison tool. Generally attributed to \cite{Wright1921}, 
$R^2$ is purely a measure of goodness of fit and is given by:
\begin{equation}
  R^2 = 1 - \frac{SS_{res}}{SS_{tot}}
\end{equation}
Where $SS_{res}$ is the sum of the squared residuals of the model and $SS_{tot}$ is the total sum of squares. A score of 1 is a 'perfect' fit
and a negative score is considered a worse first than a straight, horizontal line through the mean as a model. Similar to AIC, $R^2$ is susceptible
to overfitting as the addition of a new parameter will always improve the score. Fortunately, adjusted $R^2$, $\bar{R}^2$:
\begin{equation}
  R^2 = 1 - \frac{VAR_{res}}{VAR_{tot}}
\end{equation}
Where $VAR_{res} = SS_{res}/n$ and $VAR_{tot} = SS_{tot}/n$, only improves its score if an additional parameter improves the model more than would
be expected by chance, making it less susceptible to overfitting and a better comparative tool for this study.


\section{Results}
The cubic polynomial may have no biological underpinning but is sacrificing realism much different to sacrificing precision \citep{Levins1966}?





\section{Discussion}



\bibliographystyle{agsm}
\bibliography{Miniproject}



\end{document}